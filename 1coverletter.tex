%!TEX TS-program = xelatex
%!TEX encoding = UTF-8 Unicode
% Awesome CV LaTeX Template for Cover Letter
%
% This template has been downloaded from:
% https://github.com/posquit0/Awesome-CV
%
% Authors:
% Claud D. Park <posquit0.bj@gmail.com>
% Lars Richter <mail@ayeks.de>
%
% Template license:
% CC BY-SA 4.0 (https://creativecommons.org/licenses/by-sa/4.0/)
%


%-------------------------------------------------------------------------------
% CONFIGURATIONS
%-------------------------------------------------------------------------------
% A4 paper size by default, use 'letterpaper' for US letter
%Code for A4 paper size 'a4paper'
\documentclass[11pt, a4paper]{awesome-cv}
\usepackage{lipsum}
\usepackage{indentfirst}

% Configure page margins with geometry
\geometry{left=1.4cm, top=.8cm, right=1.4cm, bottom=1.8cm, footskip=.5cm}

% Specify the location of the included fonts
\fontdir[fonts/]

% Color for highlights
% Awesome Colors: awesome-emerald, awesome-skyblue, awesome-red, awesome-pink, awesome-orange
%                 awesome-nephritis, awesome-concrete, awesome-darknight
\colorlet{awesome}{awesome-red}
% Uncomment if you would like to specify your own color
\definecolor{awesome}{HTML}{4F2683}

% Colors for text
% Uncomment if you would like to specify your own color
\definecolor{darktext}{HTML}{000000} %Bold Subsections & Quote
\definecolor{text}{HTML}{4F2683} %Highlighted Name
\definecolor{graytext}{HTML}{000000} %Text Information
\definecolor{lighttext}{HTML}{807F83} %Address 

%%%%%%%%%%%%%University Application


% Western Hex Colour 4F2683 (Purple)
% Western Hex Colour 807F83 (Grey)



% \definecolor{darktext}{HTML}{414141}
% \definecolor{text}{HTML}{333333}
% \definecolor{graytext}{HTML}{5D5D5D}
% \definecolor{lighttext}{HTML}{999999}

% Set false if you don't want to highlight section with awesome color
\setbool{acvSectionColorHighlight}{true}

% If you would like to change the social information separator from a pipe (|) to something else
\renewcommand{\acvHeaderSocialSep}{\quad\textbar\quad}


%-------------------------------------------------------------------------------
%	PERSONAL INFORMATION
%	Comment any of the lines below if they are not required
%-------------------------------------------------------------------------------
% Available options: circle|rectangle,edge/noedge,left/right
% \photo[circle,noedge,left]{profile1}
% \photo[circle,noedge,left]{profile2}
% \photo[circle,noedge,left]{profile3}
\name{Tevin}{Heath}
\position{Senator{\enskip\cdotp\enskip}Undergraduate Representative{\enskip\cdotp\enskip}Peer Mentor}
\address{Unit 10 - 350 Dundas Street South\\ Cambridge, Ontario}

\mobile{(+1) XXX-XXX-XXXX}
\email{XXXX@mcmaster.ca}
%\homepage{www.posquit0.com}
%\github{posquit0}
\linkedin{tevin-heath}
% \gitlab{gitlab-id}
% \stackoverflow{SO-id}{SO-name}
% \twitter{@twit}
% \skype{skype-id}
% \reddit{reddit-id}
% \extrainfo{extra informations}

\quote{``You can’t Spell \textbf{Community} Without \textbf{Unity}"}


%-------------------------------------------------------------------------------
%	LETTER INFORMATION
%	All of the below lines must be filled out
%-------------------------------------------------------------------------------
% The company being applied to
\recipient
  {School of Graduate and Postdoctoral Studies at Western University}
  {Western University\\International and Graduate Affairs Building, Room 1N07\\1151 Richmond Street\\London, Ontario, Canada \\School of Graduate and Postdoctoral Studies}
% The date on the letter, default is the date of compilation
\letterdate{\today}
% The title of the letter
\lettertitle{Western University's - Masters of Public Administration (Local Government)}
% How the letter is opened
\letteropening{Dear Western University's Graduate School Admission Committee,}
% How the letter is closed
\letterclosing{Sincerely,}
% Any enclosures with the letter
%\letterenclosure[Attached]{Cover Letter Attached}



%-------------------------------------------------------------------------------
\begin{document}

% Print the header with above personal informations
% Give optional argument to change alignment(C: center, L: left, R: right)
\makecvheader[C]

% Print the footer with 3 arguments(<left>, <center>, <right>)
% Leave any of these blank if they are not needed
\makecvfooter
  {}
  {Tevin Heath~~~·~~~Statement of Intent}
  {\thepage}

% Print the title with above letter informations
\makelettertitle

%-------------------------------------------------------------------------------
%	LETTER CONTENT
%-------------------------------------------------------------------------------
\begin{cvletter}
%\doublespacing %removed the doublespacing as it is not a word limit but page limit
\setlength\parindent{24pt}


My name is Tevin Heath, I am a 4th year Bachelor of Science (Hon.) in Mathematics and Statistics at McMaster University graduating in Spring 2020. I am writing this letter applying for admission into Western University’s Masters of Public Administration (MPA - Local Government) full-time program within the Department of Political Science under the Faculty of Social Sciences. Before I begin with my Statement of Intent, I would like to address my performance reflected on my transcript during the Fall/Winter 2013/2014 as I think it is important to start with a concrete idea of the person that I am as an applicant from my transparent academic history. Before I became the student-leader that I am today, that is reflected in my transcript, references, and CV. 


As previously mentioned, my name is Tevin Heath and as provided in my transcript, I started my undergraduate degree in Humanities 7 years ago, during the academic term of Fall/Winter 2013/2014. At that time, I started my first undergraduate experience after an extra semester of high-school beyond grade 12 (i.e Victory Lap/Super 12). Where my academic journey began at McMaster University. It was challenging, not because I was not mature or ready, but I was in a program that was not a great fit for me at the time. As I became more developed over the years it was clear to me that I am strategic, analytical, collaborative, and inquisitive by nature which made the Mathematical Sciences a better choice for me. First-year Humanities was not my first choice, but I was not able to be accepted into McMaster University's Mathematics \& Statistics program straight out of high school, due to admission averages. But, in my heart I wanted to accomplish this objective, which is where my unwavering resoluteness started, I wanted to graduate with a degree in Mathematics and Statistics despite the length it might take. That first year at McMaster University 7 years ago within the Faculty of Humanities, despite attempting to use the resources available, was challenging and led to failure in my first year. Although at the time it was extremely disappointing, I did not give up easily and I made an internal quest towards graduating at McMaster University within their program of Mathematics and Statistics, regardless of how long it may take. But, most importantly during my academic journey it has aided me with a sparked determination to help, mentor, guide, and empower fellow students away from the same mistake that I made in the past. This is often reflected and shown in my many professional, leadership and volunteer activities. The focus of being a strong representative and advocate remains to be one of the main core values in my life. After that failure of an undergraduate year at McMaster University, I took a year off enrolling in continuing education in Mathematics to ensure that I remain academically relevant. Following the 1-year academic leave of absence, I was accepted into York University's Pure mathematics program where I was given a second chance to fix my mistakes of the past. This is shown from my unofficial transcript provided from York University during the Fall/Winter 2015/2016 year. I used the second chance at York University to focus primarily on academic studies to prove to myself that I am capable of success at the undergraduate level. Additionally, I used the opportunity to volunteer and give back as a representative, this will be a common theme of my life experience. After 1 successful year at York University, I was accepted as a transfer student into McMaster University's Mathematics \& Statistics program and continued as a soon-to-be graduate in Spring 2020. Which ultimately completed a 7-year goal set out regardless of the challenges and failures that were made in the past. In the next section, I will answer the topic for my Statement of Intent that reflects why I still believe that I would be a great asset in the Masters of Public Administration Program (Local Government) as a full-time student.

\newpage
I would like to start by mentioning that I am extremely grateful to have reached out to alumni and current students of the Western's MPA program for consultation, discussion, and their experiences, despite being currently enrolled as an undergraduate student. From these discussions, it became more clear that the fit of this program would be best suited in my career trajectory. I am grateful for the opportunity to meet full-time students and part-time working professionals, and one of which is currently a student that is from the City of Cambridge. During many of these discussions and my research, I discovered that Western's MPA program has a high emphasis beyond local government, and towards areas such as professionalism, community, and mentorship. This is shown from a wide range of courses, opportunities, and activities such as the alumni conferences where alumnus would meet students, build networks and connect, to the MPA's mentorship program where alumni or working professionals in the part-time program are paired with full-time students towards fostering a mentor-mentee relationship for life-long advice. This high emphasis on community and focus on successful local government job outcomes after graduation is one of the reasons why I am most excited to apply. As the quote above states, “you can't spell community without unity”. Although, there are many reasons why I am excited to be applying another minor compelling reason for me personally is simply because of the location. Currently, my main residence resides in the City of Cambridge. As Western is located in the City of London, it is a manageable commute for me home, if necessary. The location is important because I would like to continue and build my network within the City of Cambridge and design a final research project that focuses on both an immediate and persistent problem in Cambridge while using the courses throughout the MPA program to aid in my administrative, strategic, and public service development. I believe this program would allow me to keep my existing connections and develop a planning study that can be both tangible and meaningful. Finally, I am interested in learning more about specific faculty member's research and advice that relate to my career trajectories. Specifically, Dr. Joseph Lyons and his thoughts on special-purpose bodies and how I can target my career end goal in that area after being accepted and graduating from this program. I place a lot of emphasis on community and being a role model and leader within the City of Cambridge and in the following section, I will go into more detail about why my experiences as a student-leader of mathematics and statistics would make a great fit for the Master of Public Administration program at Western University.

When I began at McMaster University within the Math \& Stats department in the Fall of 2016/2017 after 1 year of Pure Mathematics studies at York University. I became determined to become a student-leader specializing in professionalism, mentorship, and strategy throughout my studies. As a result, I focused on becoming a holistic student-leader that continues to learn and grow in all my coursework, by choosing to develop as a strong assertive communicator and advocate, a compassionate role model who is a team player and consultative for friends, fellow students, and constituents that I may represent or work with,  and to be a great mentor with a mature and poised personality on issues facing undergraduate experiences. These issues faced by undergraduate students are vast and frequent such as emotional \& mental health-related. While ensuring that I give as much effort into my coursework and assignments. I may not be the most academically gifted amongst my graduating class, but I am driven, professional, consultative, poise, and continuously improving and when it came to electing someone to represent the voices of undergraduate student health, policy, or barriers that affect the Department of Mathematics and Statistics, The Faculty of Science, or McMaster University as a whole. I am often selected to be their undergraduate champion. As an undergraduate student, I have been faced with many diverse courses, but to enroll and begin as a graduate student in the MPA program, it would suit me best since I will only be studying what I am most passionate about and would like to learn with the purpose to apply in my career after graduation. As a result, from my experience as a representative, the logical and personal step for me would be to allow my professional, strategic, and administrative development to continue and extend within the Masters of Public Administration with a strong emphasis on local government, as it relates more closely to what I am currently facing and where I see myself in 5+ years after graduation. And for these reasons, I confidently believe that I would be a great asset and a valued full-time student within Western’s MPA (Local Government) program for which I hope to bring a unique perspective in my coursework, discussions, and networking interactions.

In the next section, I will go into further detail about my career goals, specific areas of the program that I am interested in, and how I envision my experience in this program that will help me accomplish them. However, I would like to mention that I am a non-traditional student, with a breadth of leadership experience. As mentioned to me from a part-time student within the program, they mentioned that it is okay to be a non-traditional student. As such, the study of Mathematics and Statistics although it does not teach the fundamentals of policy analysis, evaluations, or public policy instead it trains us on how to remain objective, critical, and logical in our thinking, coupled with evidence-based statistical analysis, techniques, and tools. I believe this gives me the freedom to grow a skill set that is distinct within the Master of Public Administration program. Throughout my undergraduate studies, I ensured to develop a diverse range of knowledge from different academic disciplines. Also, utilizing my professional and leadership experiences of being an interdependent and collaborative learner. It has allowed me to balance both the challenge technical skills from my coursework with, the leadership and communication skills from my professional, leadership and volunteer activities.

\newpage
Now, this brings me to the present, my career goals, specific areas of the program that I am most excited about, and how I envision this program will help me accomplish these goals and set me up for success after graduation. First, I would like to mention that I am excited \& honored to be applying. My career goal over the past 5 to 10+ years is to become a Municipal Strategist \& Director within the City of Cambridge. Currently, I plan on getting involved with advisory committees as a voluntary member within the City of Cambridge and the City Hall. I believe this career goal has two parts, first for a short-term goal after graduating, I would like to start my career journey towards being a Municipal Strategist within either special-purpose bodies such as local public health units by aiding the health policy design process and strategic health planning \& management. Or, by getting more involved with the City of Cambridge's Community Well-being Advisory Committee. And secondly, long-term I would like to build a community-based venture within the City of Cambridge, from my experiences on the Community Well-Being Advisory Committee that collaborates and develops relationships with City Hall and the Mayor, Elected Council Members or the Region of Waterloo Public Health on identifying community health needs and concerns affecting the regional areas within the City of Cambridge such as Galt, Hespeler, and Preston. Additionally, I would like to learn to lead, manage, and design policies, solutions and creating partnerships to investigate how the City of Cambridge responds to poverty, homelessness, mental health, addiction, and public safety. I would like to start this venture during the MPA program as well as build upon tools and skills in the MPA program and utilize the mentors, networks, students, coursework, advice and career goals to aid in this progress and develop as an emerging strategist, consultant, and director. 


At McMaster University's Department of Mathematics and Statistics, an emphasis is placed on mathematical and statistical theory. However, as a holistic student-leader, I need to use those skills of statistics, modeling, and data science and extend them in the policy-making and discussions for immediate and sustainable impact. As such, I would like to extend it towards a more applied route of policy design, formulation, decision-making, implementation, and evaluation by collecting and disseminating evidence-based research to aid in local government concerns in healthcare public health issues, or related areas in local government. Currently, I do not know these issues but I believe within the program I can learn to understand them and develop ways to determine these tools of issue spotting evaluations, from courses such as PA 9914 - Research Design and Methods and PA 9915 - Program Evaluations (Instructed by Dr. Bill Irwin). Additionally, through my winter 2020 coursework, I am positioning myself in my last year of Mathematics and Statistics to complete courses that would best prepare me for graduate studies in public administration. For example, selected courses such as accounting information and decision-making, personal finance, data science theory, and public service leadership. At the moment from many of my professional and leadership activities that I am involved in at McMaster University such as their Budget and Planning Committee, it continues to provide me with a clear understanding on how a large educational institutions such as McMaster University strategies and constructive dialogue and discussions about how high-level perspectives are implemented within McMaster University as a whole. But, I believe there is more educational growth to ensure that I am being trained in ways that remain relevant within my career goals and propel me in local government excellence. From the MPA program, there are two other courses that I am interested in being enrolled in are PA 9923 - Strategic Planning and Management and PA 9917 - Issues in Local Government (formerly instructed by Dr. Joseph Lyons - Summer 2019). For PA 9923, I am excited to go more in-depth on the theory of strategy, planning, and management that will allow me to develop concrete understandings of my current experiences from executive committee boards. For PA 9917 as this course is offered near the end of the program whereas a graduating class we would have to present our final report. I am interested in learning about the feedback provided and listening to unique topics that classmates in the program are tackling. I think this part of the program coupled with the previous courses of instruction will synergize beautifully bringing all the experiences together throughout the program. Although there are more areas I am interested in, I am most excited to be part of a great learning community of both part-time working professionals and full-time students that all have a focus on local/municipal government.

Thank you for the opportunity to consider my Statement of Intent and my application to be enrolled in Western University's Masters of Public Administration (Local Government) Program. And, I would like to re-emphasize that I am highly looking forward to being a full-time student and sharing my experiences as a dedicated student throughout the program and a continuous mentor after graduation. But most importantly I will be using my time between coursework, projects, and conferences to make myself an ideal leader and role-model to give back to the City of Cambridge, “The People, A Place, the Prosperity”.
\end{cvletter}


%-------------------------------------------------------------------------------
% Print the signature and enclosures with above letter informations
\makeletterclosing

\end{document}
